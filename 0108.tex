\subsection{体育节当天}

六月上旬——流汗的季节渐渐来临,周的学校在此时举办了体育节。\\

比起小学、初中那种和和气气的运动会,高中体育节的气氛更像是课堂的拓展,几乎不会有家长前来观看。\\

即使如此,毕竟体育节也是为数不多的活动,还是有一部分学生热情高涨着。特别是运动社团的低年级学生尤其干劲十足。或许他们是觉得,这是向社团顾问展现自己能力的机会吧。\\

相反,文化社团的学生则有很多没什么兴趣。

不参加社团的周属于后者。\\

「好累啊」\\

与周在同一个帐篷里的学生小声说道。听到他这么说,周暗自苦笑起来。\\

虽然提不起兴趣,但周也不至于明摆出没干劲的表情,于是他就装出一副假装没听见的样子。\\

幸运的是,他报名的第一志愿顺利通过了,所以他不需要在那些跑来跑去的项目里出场。对他来说需要跑来跑去的,顶多也就是全体男生参加的骑马战了。\\

「藤宫看上去不怎么讨厌啊,我还以为你会讨厌」\\

同在红组帐篷里的门胁惊讶地看着周的脸。\\

「反正想报名的通过了,现在也闲着,所以这回没那么讨厌。虽然我觉得学习更加轻松就是了」

「我觉得你这样也很少见……」

「藤宫学习优秀,但是运动不太行嘛,没办法」\\

周没法否定在旁边听着对话的柊,露出了苦笑。\\

毕竟事实就是这样,周无从否定。然而,被人指出这一点,还是会让他产生复杂的心境。\\

周当然很感谢学习优秀的评价,也为别人眼中的自己是这样而感慨,但他禁不住会对文武双全有所憧憬。\\

「是不是该定期去运动一下比较好啊。我姑且有在散步或者随便跑跑」

「要是家离得再近一点,我倒是可以和藤宫一起慢跑」

「我哪跟得上门胁的速度和体力」

「就是说啊优太。你是不是忘了我上次那么一来,结果差点要死了。你那不叫慢跑,就是跑步」\\

九重似乎是陪门胁慢跑过,一副没劲的表情。\\

顺带一提,九重不是体育社团的人,而是在文化社团,记得是在天文社里。他身材瘦长而纤细,不但个子小,皮肤还很白,怎么看都不像是会运动的样子。

虽然这么说,不过纤细小个子的真昼倒是样样运动都能做,也不能一概而论。\\

「不,我觉得藤宫能行的。马拉松那会儿,看你也不是特别累」

「为了年老的时候考虑,我每天都会锻炼体力,但比不过搞体育的人啊」

「也只有你会现在就会考虑老后的事情……」

「藤宫真是奇怪啊。啊不,该说是有趣比较好?」

「你那是在夸我吗」\\

柊这人性格正直而诚实,说话也很直……他的直截了当、不客气,都是周刚刚开始和他相处时就已理解的事情。\\

「一哉他是在夸你哦,大概」

「那就谢谢了」

「不用谢」

「这段对话是在搞什么……」\\

九重毫不掩饰自己的傻眼,不过其中没有嘲笑,只是单纯的傻眼而已。

他的表情中也能看到一丝欣慰,所以应该不只是表面上的意思。\\

「算了算了,一哉一直都是那么天然」

「我觉得我不天然啊……」

「所谓的当局者迷嘛。没事,一哉你不用在意,保持你的本色就好」

「唔,是吗?」\\

柊轻易地接受了,没有再去追问。周小声说了句「这样也可以吗……」同时看向操场那边。\\

操场上,选手们正在短跑。\\

从赛道的长度来看,应该是100米跑。第一批已经跑完,第二批开始排队了。\\

第二批似乎是女生队伍,里面聚集着我方跑步比较快的女生。

其中还有一名眼熟的、棕红发色的少女。\\

「咦,千岁跑得快吗」

「嗯,很快。白河初中可是田径社的」

「哇,这样吗」

「嗯。不过高中她没进,说是和社团里的学长学姐发生争执很麻烦」

「我该吐槽她把发生争执这事作为前提吗」

「不是,那个,其中有点缘由……总之,就是她吸取了教训,或者说是累了吧」

「……累了?」

「白河和树的交往中间有很多曲折。该怎么说呢,嗯,田径社有个学姐喜欢树」

「啊,我懂了」\\

现在两人是全年级都认可的情侣,但听当事人说,在初中交往之前,是树一直在对千岁猛烈进攻。

据说,当时千岁的性格比现在要冷淡一些,树花了很长的时间攻略她,最后才得以交往。\\

若是喜欢树的社团学姐看到那一幕,也就不难想象会产生争执。\\

「因为这种包袱很麻烦,所以她就没加入社团。不过,她还是挺喜欢跑步的,偶尔能看到她跑」\\

门胁笑着补充了一句「毕竟家离得近」,然后看向了摆出蹲踞式起跑姿势的千岁。\\

就算周几乎是个外行,但他也看得出千岁的姿势很精湛,甚至让人觉得漂亮。

从远处看,她的表情不像平时那种胡闹的、无忧无虑的笑容,而是一本正经、严肃认真。\\

发令枪响遍操场。\\

这一瞬间,千岁是最快行动的。\\

她以任何人看了都会直呼漂亮的姿势冲了出去,疾驰如风,甚至甩开了现任的田径部女生。

她柔软的秀发被甩到后方;她的身体笔直地往前冲去;她使劲瞪地的脚,比其他选手更快地奔向终点。\\

千岁跑步的样子美丽得让人出神,再等回过神来,她已然越过了终点线。\\

率先跑完全程后,千岁拿着第一名的旗子,看向红组……也就是看向周这边,乐呵呵地笑着。

她那满足地猛挥旗子的模样,甚至令人欣慰。\\

跑完100米回来后,千岁自豪地挺起胸膛。\\

「我回来啦~看到了吗?」

「看到了看到了。好快啊」

「哇~谢谢~!」

「是啊。白河跑步的样子看着就很舒服」\\

得到现任田径社社员的两人表扬,千岁心情大好。周也称赞道「辛苦了,跑得真快」。

实际上,她快得出乎意料,周都吓到了。不过千岁倒是没有非怎么样不可的气势,只是憨憨笑道「啊~好开心」。\\

到底是千岁,这种毫无紧张的感觉和跑步时截然不同。周也放心下来,露出弛缓的笑容。\\

「话说回来,白河还是一如既往地快啊」

「嘿嘿~毕竟每天都在练嘛。虽说肯定是没有社团那时快了」\\

看来千岁初中时比现在更快,真令人吃惊。在周的周围,很多人都在身体、头脑等方面有一技之长。周作为平凡的人,感到无比羡慕。\\

虽然柊似乎也和门胁他们一个初中,但他也还是对千岁没有加入田径社却有这个速度而感到吃惊。\\

「我一直都在想,你怎么会这么快,是因为表面积小,能减少空气阻力吗」

「哉儿哟,表面积是说哪儿啊」

「嗯?是说身高来着」\\

柊用纯洁的眼神看着千岁,好像在说「除此之外还能是什么」。

千岁皱起了眉头。她这表情与其说是生气,不如说是对自己感到羞耻吧,她准以为柊是在说体格了。\\

顺带一提,千岁虽然个子不像真昼这样小,但也不能算高。

从女生的平均水平来看,她算是偏高的,而作为田径选手又算不上多高。\\

除此之外,她还体型瘦长,不像是运动员,或许正因如此,柊才会吃惊于她的速度。\\

就柊的表现来看,感觉不出他有什么别的意思。所以,这完全是千岁误会了。\\

「丢人了吧白河」

「诚儿你好吵」\\

千岁刷地脸上一红,啪啪拍着九重的后背,同时坐到旁边。周则以不让她发现的方式,露出了轻轻的苦笑。
