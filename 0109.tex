\subsection{天使大人的借物}

周要参加的项目,数起来也只有投球、借物赛跑,再加上所有男生都要参加的骑马战这三样,相对来说还挺闲的。

有些激情满满的学生报了不止两个项目,但周对体育节不那么上心,于是只报名了两个项目和团体竞技而已。\\

顺带一提,投球已经结束了。\\

这种竞技热烈不起来,说穿了不过就是把球扔往高挂着的篮筐里而已。\\

虽说投进去的球需要抢,但本来球就有很多,也没必要去较劲,因而整个比赛自始至终都很和平。\\

出场之前,千岁还叫他多表现表现,问题是投球也没什么可表现的。

投球不过是简单地捡几个球,让它们滚向一处收集起来再投出去,如此这般枯燥任务的重复,没有什么吸引眼球的。\\

唯一值得一提的是,他们最后投进的球数比白队要多,这大概是投得比较准,再加上把球集在一处起了作用的缘故。\\

「我说周,你还真是净捡些不显眼的项目啊」

「啰嗦。倒是你差不多要换班了吧,还不快去」

「啊,是哦」\\

千岁看着自己的日程表,一边嘟哝着「执行委员还真是忙啊~」一边朝着运营组的帐篷动身走去。

周有一点「那你当时为什么要报名」的想法,不过现在这也都是马后炮了。\\

他一边望着千岁一路小跑的背影,一边浏览着贴在帐篷柱子上的今日日程。\\

上午的安排只剩下几个项目,其中也包含了周最后出场的个人项目——借物赛跑。

等这几个项目都结束,过了午休,便是下午的日程。\\

总而言之,借物赛跑结束之后,周就只剩下下午的骑马战了。\\

「……话说,借物的时候是那家伙负责运营啊」\\

这时候换班就说明,接下来的项目恐怕是千岁负责,借物赛跑的裁判员也一定会是她……周甚至觉得她就是看准了这一点。

他不知道这回借物的题目是谁想的,估计就不是什么正经题目,感觉有点吓人。\\

周的心情变得有点沉重,不过,他还是朝着下下个项目,也就是借物赛跑的集合点走去。真昼似乎也报上了这个项目,正静静地待在那里。

反正也没有要说的事情,周就没有开口,但和真昼对上眼的时候,她淡淡地微笑着向周点头致意。\\

虽然在外面两人还是保持着一般人的距离,但周望见真昼微微流露出平日里的笑容,还是有点心跳加速。

周也同样面无表情地向真昼致意,不过总感觉心里有点不自在。\\

而负责体育节运营的千岁,在召集选手之后,则是愉快地望着这两人。\\

\vspace{2\baselineskip}

到了借物赛跑的时间,选手们听从负责人……也就是千岁的指令,进入了操场。

操场上散落着大量叠起来的纸片。接下来要做的就只是合着比赛开始的信号,去捡起那些纸并照着里面写的题目把东西拿来而已。\\

借物赛跑与其他的跑步项目不同,属于比较休闲的项目,目的也是享受借物的过程,因而不用那么较真。

不过根据拿到的题目,也有可能落得公开处刑的下场,这一点还是要小心。\\

「各位选手,请站到起跑线」\\

千岁用话筒发出了清晰的指示。只要不胡闹,她就确实适合当主持人:不但性格明朗,善于察觉气氛和状况,而且嗓音也很清澈,不会太尖,容易听清楚,足够用来集中他人的注意了。\\

由于全校的学生和职员都在看着,目前千岁没有一点胡闹的样子,发出了「各就各位」的信号。

不过,发令枪是由另一个负责的男生拿着,千岁负责的也只是倒数。\\

在千岁喊出「预备」口令后过了一拍,发令枪响起了砰的一声。\\

虽然这枪声不管什么时候都对心脏很不友好,但周不动声色地慢跑向放着纸的地方。

跑得快的选手已经展开了纸片,浏览着里面的题目。周跟在他们后面,捡起一张叠起的纸确认内容。\\

纸上工工整整地写着几个字。\\

『你认为漂亮的人』\\

他手上的题目要借的不是物品而是人,这种情况周也有料想到。\\

虽然周很想吐槽到底是谁想出来的题目,但幸好这题他还勉强能解决。

好在题目不是最难搞的『喜欢的人』这种类型,那么只要带来一个客观上看起来很漂亮的人就好了。\\

也就是说,把大家公认的美人——真昼叫来就行了。等真昼借来东西之后,跟她一起冲过终点便是。\\

虽说带着真昼走恐怕会很显眼,但毕竟这是题目要求,等大家知道题目之后,应该也会理解这是个恰当的选择吧。\\

周这么思考着,正打算去找大概正在捡纸片的真昼——结果,有人从一旁抓住了他的T恤。\\

准确来说,这人的动作并不是抓,而是捏。周的衣摆被人轻轻地拉了几下,他疑惑地扭过了头。\\

在周的眼前,他在找的人正客气地看着他。\\

「藤宫,我要借的是你,等你借到了要借的东西之后,可以让我也一起跟着吗?」

「咦,我吗?」

「是的」\\

居然互相是对方要借的东西,这让周始料未及。\\

虽然某种意义上也算是正好,但他觉得这样会非常显眼。

不过,在大操场的正当中,从真昼来找他搭话开始,再谈什么显眼不显眼就已经没意义了。\\

在终点线的对面,负责裁判的千岁正一脸坏笑地看着他这边。\\

(好家伙,给我记住了)\\

说到底,纸上的字就是千岁写的,这题目肯定含着一部分她的小算盘。虽然不知道真昼抽到了什么题目,但从真昼特意要选择周来看,那题目肯定是真昼不能让步的。\\

「啊……话说你要借的是什么?」

「保密」\\

明明过了终点线就要宣读出来,真昼却不肯把题目说出口。

于是,周只好叹了口气,然后向终点前进。\\

「我要借的东西也正好是你,那就一起去终点吧」

「……藤宫要借的又是什么?」

「保密」\\

周照着真昼的回答原样奉还后,真昼微微笑了出来。\\

「嗯,那就等到冲过终点线再揭晓吧」\\

真昼轻轻说完,牵起了周的手。

她不理会四周的嘈杂,触碰着周朝终点前进。\\

虽说周感觉稍有些胃痛,但看见真昼满心欢喜的样子,就会觉得这些事都是没办法的。他也因而认识到,喜欢上对方就会处于弱势。\\

穿过这个让周觉得有些待不下去的操场,两人抵达了终点。在那里,心情大好的千岁正等着迎接两人。

周不禁啧了一声,而千岁却毫不介意。\\

「哎呀,这是两位一起冲线嘛~?没记错的话你们两位都是参赛选手」

「好你小子一脸坏笑的。我们互相是对方要借的东西」

「哦嗬~那么就来确认下题目吧,你们哪位先来?」

「请让藤宫先来」\\

真昼不假思索的答复令周吃了一惊,可千岁像是在说「了解」一样,指向了周拿着的纸,意思应该是叫周给她看吧。\\

周想着,反正也不是非得藏着掖着,便痛快地转向千岁,给她看了题目。

看完题目的内容,千岁露出了略微失望的表情。

不知道千岁在期待什么,看起来这并不是她想要的结果。\\

尽管如此,千岁还是恢复了气势,满脸堆笑地把话筒拿到嘴边。\\

「现在开始检查题目。红组第一拿到的题目是——『你认为漂亮的人』」\\

听到读出来的题目,众人散发出一股安心的气氛。\\

周的选择十分稳妥。在他所知的范围中,学校里没有人比真昼更漂亮,况且周确实觉得真昼是最可爱的。

就算抛开周个人的看法,把真昼带过来也是完全正常的选择。\\

虽说和真昼一起到达终点势必引来旁人的敌意,但题目内容如此,这种敌意多少有所缓和。\\

问题就在真昼这边的题目了。\\

虽然周不知纸上写了什么,但想到真昼特意指定了自己,他便情不自禁地觉得,真昼的题目会让自己安宁的学生生活不保。\\

千岁从真昼手上接过写着题目的纸,眨了眨眼,然后观察起了真昼。

虽然从周的角度看不见纸上写了什么,但千岁的表情似乎是在说着「我可以说出来吧?」一样。\\

(到底是拿到了什么题才要把我带来啊)\\

千岁的反应更加让周摸不着头脑。

真昼的脸上却依旧是安宁的微笑。也就是说,即使读出来也没有关系。\\

千岁确认了真昼的意向,便恢复了平时的笑容。\\

「嗯~那么来看一看同时冲线的白组第一拿到的题目。白组第一拿到的题目是——『重要的人』」\\

在千岁的声音响彻操场的同时,学生休息处迅速响起了一阵喧嚣。\\

周下意识地看向真昼——她与周对上了眼神,淡红色的嘴唇描出一道弧线。\\

那表情,既像是小孩子成功恶作剧时的笑容,又透出了几分羞涩。

可以确定的是,真昼看向这边,是为了看周知道题目内容时是什么反应。\\

(好一个小恶魔……)\\

思维缜密的真昼很容易就能料想得到,要是把题目公开,周围的反应会是如何。\\

即便如此,真昼还是决定选择周作为自己借的东西——为了给两人的关系带来变化。

从今往后,两人将不再是模棱两可的外人。\\

真昼展现出的,不是平日在学校里露出的美丽笑容,而是朝周露出的真心微笑。周低声叹道「这之后肯定要被周围的人问东问西了」,胡乱地挠了挠头。
