\subsection{天使大人与传言}

黄金周结束,又到了上学的时间。周刚进教室就听到里面吵吵闹闹的,从而倍感困惑。\\

放假回来,同学们会聊假期发生的事情,热闹也是常有的事。然而,今天的喧嚣却和那样的热闹有所不同。

现在这种热闹,与其说是在讲假期发生了什么,更像是在谈论什么传言一样。\\

周把东西放到座位上,竖起了耳朵——那些同学好像是在说真昼的事情。\\

「听说椎名前几天跟一个帅哥约会了」

「是年初传出来的那个人吗」

「她说没有在交往,不过……」\\

听到「约会」这个单词,周就脸颊抽搐。\\

周虽然一定程度上预料到了会遭到目击,但他完全没想到这个话题会席卷整个教室。\\

这些女生一边悄悄讲着传言,一边看向真昼那边。

真昼好像没有注意到——或者说是注意到了但当作耳旁风——正在为第一节课做着准备。\\

尽管她凛然清秀的样子一直都备受瞩目,但今天有很多人看向她的眼神中还混着些好奇。\\

「我、我说椎名」\\

那群人中的一名女学生,好像下定了决心一样朝真昼搭话说道。\\

「那个,前几天,我看到椎名和一个男人在商场里逛」

「嗯,是有这回事」

「是在和那个人约会吗?」\\

周一邊想著「敢去正面问,是勇士」一邊感到担心,而真昼只是连连眨眼。\\

看那样子,应该是会用往常的天使大人那种应对吧。\\

是的。如果是平时的真昼,应该会用天使的笑容坚决否定才对。\\

「是啊。考虑约会的定义的话,算是约会吧」\\

不知真昼在想什么,她竟然肯定了。\\

约会的定义,基本上是男性女性决定好时间见面。这么一说是没错……但她们不会去考虑约会的严格意义吧。\\

周围发出「呀」的尖叫。

无论什么时候,女性总是喜欢热烈讨论其他人的八卦。如果是平时,周还能远远望着她们,感叹着「真能讲啊」轻轻带过,但这次他作为当事人,自然不能这样。\\

「椎名在和那个人交往吗?第一次听说!」

「没有交往,但他是我最重要的人」\\

伴随着尖叫声,周开始微妙地胃疼。

真昼完全没有撒谎。周也知道她很重视自己。然而,她也不是在讲述真实情况,而是在说着怎么解释都可以的内容。\\

一旦真昼这么做,肯定会流出她喜欢上那个男人的传言。但真昼只是微笑注视着这些兴致高涨的女生。

她往周的方向瞄了一眼,露出了并非天使大人的那种笑容。周扶住额头,又高兴,又害羞,又胃疼。\\

树似乎是同样在听着。他露出坏笑,拍了拍周的肩膀,而周只觉得身体一口气就积累起了疲劳。\\

\vspace{2\baselineskip}

「老实说,我觉得那已经没法彻底否定了」\\

回家路上,两人顺路去了趟快餐店。在店里,树像是想起什么似的开口说道。\\

由于下午的体育课,周明明吃过午饭,肚子却又饿了。现在,他们正在边吃薯条边聊着天。树或许是想起了早上的骚动,露出苦笑。

另外,今天很不太平,周就没能和真昼一起吃饭。真昼一直在遭到众人的质问,周实在是无法去和她坐到同一桌。\\

「没法完全否定?」\\

树的意思是大概是「没法像以前一样否定」,但周却不知道这是为什么。\\

「我说啊,优太光是看到都知道你们关系亲密了。你们走在路上肯定是牵着手贴在一起的吧」

「唔」

「让人目击到这种状况,否定交往也说不通,她也就只能这么说了吧」\\

这么一说,周完全无法反驳。

确实,前些天的出游,客观来看就像是亲密的约会一样。两人有时牵手、有时靠在一起、有时欢笑,在不知情的人眼里看上去毫无疑问就是情侣了。\\

所以,真昼的应对也是没办法的……吧。\\

「而且那应该也是种表示吧」

「表示?」

「为了驱赶男生,还有为了表现自己没有和女生敌对的意思?优太也是一样,受欢迎之后多多少少会有人嫉妒。那个天使大人也是一样。只要她暗示有喜欢的人,不把其他人放在眼里的话,哪怕她跟优太之类的人在一起待一会儿,也不会让别人以为她对优太有意思」

「……原来如此」\\

周不是受欢迎的那群人,没多少这种感觉。不过,不管人气多高,总还是会有极少数的人没法接受。要得到所有人的喜爱是不可能的。

尽管周从未见过有人公开批判真昼,但肯定有人会对她感到不满。\\

一旦受欢迎,同性就会抱有偏见,这一点似乎是男女共通的。\\

「不过她应该还有一个目的就是了。你自己想想看」\\

树说还有一个目的,但周却完全想象不出。

周不觉得在「为了让男生放弃、为了防止同性嫉妒」之外还能有什么理由。\\

「我说,周君啊」

「咋了」\\

看到周摸不着头脑的样子,树露出了无语的眼神,不过还是耸耸肩感叹了一句,结束了这一话题。\\

「你现在这样是不显眼,但好好整一整的话,即使跟那个人走在一起也不会有人说什么的。别再那么否定自己了」\\

树说出这话是想要治一治周缺乏自信的毛病。周不太情愿地点头同意了。\\

「……我知道」

「哎,不会得意忘形是你的优点,可也是你的缺点。你到底为什么那么没自信啊,你这没自信都根深蒂固了」

「要说为什么啊……就是过去发生了点事」

「那事情,告诉我会有问题吗?」

「倒不是我想藏着掖着,只是没什么好说的」\\

周并没有什么悲惨的过去,也没有遭到谁的欺负。

只是他曾相信的人,否定他,說他没有价值,仅此而已。\\

就这点小事还始终不能忘怀,周自己都觉得丢脸。然而,这事一直插在他的心头,至今仍有阵痛。

虽然这阵痛已经不会像当时那样发作,但周还是会感到空虚。\\

自从和真昼相遇,周已经不再那么受到自我厌恶的折磨了,然而拿不出自信这一点还是一如既往。\\

「那等你想说的时候,我再听」\\

树没有打算深究。\\

他就是像这样直觉很好,或者说,擅长观察对方真正不想让人涉及的界限。树负责制造班级里的氛围,正因為是他才能做到洞察感情中的微妙。\\

树轻易退下之后,看着周嘿嘿地笑了起来。\\

「总之,不管怎么说你都应该多点自信。你已经不是过去的你了。干脆换个形象脱胎换骨都行哦?」

「那就算了吧」

「真小气啊,我还期待着『使用前』和『使用后』的区别呢。算了,也行吧。在你自说自话否定可能性之前,先好好努力让她看你啊。你还不够积极」

「积极,这样吗」

「积极奔着推倒去」

「你傻吗,那怎么可能做得到」\\

周相信,要是做了这种事,肯定会被她鄙视。

这和摸头牵手是不同层面的事情。由于父母的事情,真昼本来就对这种事情很敏感。周並没有在跟她交往,而且如果信赖的人做出这种事情,必然会一下子被厌恶的吧。

这和简单的身体接触完全不同。\\

「至少也该拥抱个嘛」

「你把我当什么了……」

「没自信、拐弯抹角、胆小鬼。明明能下意识身体接触,就是沒辦法主动去碰」%感覺略怪

「唔……话说,你怎么这么确定啊,身体接触之类的」

「嗯?是小千听她说的」

(这个能不能别泄露出去啊)\\

真昼跟千岁说了多少呢。这就是为什么千岁有时會用促狹的眼神看著周吗。\\

「是男人就有点胆子。就算不表白,能牵手就能更进一步去接触吧。去缩短点距离啊」

「你说的倒是轻巧」

「我实在替你着急,都想踹你背了。她都那么敞开心扉了,只要不性骚扰,有点身体接触也没什么吧。听小千说,你摸头了吧?那拥抱这点事怎么会不行」

「别强人所难啊」\\

周说不出口「其实拥抱过」这种话。

那次是为了支持快要不行的真昼,没有什么恋爱的意思。周觉得他做不到带着喜欢的心情主动去拥抱她。说到底,真昼会答应吗。\\

「你去问问她试试看?我是觉得她会爽快同意的」

「怎么可能啦……」

「这可说不准。那个人其实還挺容許亲近的人接触她的。小千跟我说,前几天椎名住她家的时候,一起洗了澡还睡一个被窝了」\\

黄金周的时候,真昼确实是去千岁家里住过。周没想到她们关系都好到一起洗澡了。

但周觉得主要原因大概是千岁缠着真昼这么做。\\

「不过给小千抢先了,我也只能心疼心疼你了」

「我说……别拿同性朋友和我这个异性比啊」

「嗯,再怎么说小千做的那些还是不可能的吧。总之,你先去试着拥抱吧,积极一点,好不?」\\

朝着坏笑的树,周嘟哝了一句「事不关己就指点江山」,把薯条盒子里剩下的发硬的薯条屑倒进了嘴里。
